This software creates seating plan for examinations. It provides five different strategies to create seat plan.

\section*{R\-E\-Q\-U\-I\-R\-E\-M\-E\-N\-T\-S\-:}

\begin{DoxyVerb}1) GNU G++ Compiler
2) Configure public_html and cgi-bin in home
3) CGICC Library
4) MySQL Connector for C++
5) jwSMTP Library
6) exim4 mail server
7) Enable site(localhost/bakaplan) for BaKaPlan
\end{DoxyVerb}


Installation of requirements

1) G\-N\-U G++ Compiler

Run following command in terminal to install \begin{DoxyVerb}$ sudo apt-get install g++
\end{DoxyVerb}


2) Configure public\-\_\-html/cgi-\/bin folder for executing files on browser.\par
 Assuming you already installed apache if not then run following command in terminal \begin{DoxyVerb}$ sudo apt-get install apache2
\end{DoxyVerb}


{\bfseries Steps to configure public\-\_\-html} \begin{DoxyVerb}$ mkdir ~/public_html

$ sudo a2enmod userdir

$ sudo service apache2 restart
\end{DoxyVerb}


Give 755 permissions to public\-\_\-html directory \begin{DoxyVerb}$ chmod -R 755 ~/public_html
\end{DoxyVerb}


Now open \href{http://localhost/~username}{\tt http\-://localhost/$\sim$username} in browser. Here username is your login name.

{\bfseries Steps to configure cgi-\/bin in public\-\_\-html} \begin{DoxyVerb}$ sudo a2enmod cgi

$ sudo a2enmod cgid

$ sudo service apache2 restart

$ cd ~/public_html

$ mkdir cgi-bin

$ cd /etc/apache2

$ sudo vim sites-available/default
\end{DoxyVerb}


Add following text in file\-: \begin{DoxyVerb}ScriptAlias /cgi-bin/ /home/*/public_html/cgi-bin/
<Directory "/home/*/public_html/cgi-bin">
    AllowOverride None
    Options +ExecCGI -MultiViews +SymLinksIfOwnerMatch
    SetHandler cgi-script
    Order allow,deny
    Allow from all
</Directory>
\end{DoxyVerb}


Save it and then restart apache \begin{DoxyVerb}$ sudo service apache2 restart
\end{DoxyVerb}


3) C\-G\-I\-C\-C Library\par


Run following command in terminal \begin{DoxyVerb}$ sudo apt-get install libcgicc-dev
\end{DoxyVerb}


O\-R

Download any latest package from \href{http://ftp.gnu.org/gnu/cgicc/}{\tt http\-://ftp.\-gnu.\-org/gnu/cgicc/}\par


Then run following commands in terminal \begin{DoxyVerb}$ tar xzf cgicc-X.X.X.tar.gz 

$ cd cgicc-X.X.X/ 

$ ./configure --prefix=/usr 

$ make

$ sudo make install
\end{DoxyVerb}


4) My\-S\-Q\-L and My\-S\-Q\-L Connector for C++

Run following commands in terminal \begin{DoxyVerb}$ sudo apt-get install mysql-server mysql-client

$ sudo apt-get install libmysql++

$ sudo apt-get install libmysql++-dev
\end{DoxyVerb}


5) jw\-S\-M\-T\-P Library

Download \href{http://sourceforge.net/projects/jwsmtp/files/latest/download}{\tt jw\-S\-M\-T\-P} Library

Follow steps to install \begin{DoxyVerb}$ cd ~/Downloads
$ tar -xzf jwsmtp-X.X.X.tar.gz
$ cd jwsmtp-X.X.X
$ ./configure
$ make
$ sudo make install

NOTE: If you got error in */mailer.cpp or */demo2.cpp file then
include string header file in that. And again run make
\end{DoxyVerb}


6) exim4 Mail Server

Run the following commands in terminal \begin{DoxyVerb}$ sudo apt-get install exim4
$ sudo dpkg-reconfigure exim4-config
\end{DoxyVerb}


A Mail Server Configuration window will appear.\par
 Follow the following instructions to configure the mail server. \begin{DoxyVerb}1) The first page is just and introduction. Press ENTER
2) On the second page choose the second option i.e 
   mail sent by smarthost; received via SMTP or fetchmail and 
   press ENTER.
3) Next Keep the system mail name as it is and press ENTER.
4) Just Press Enter for the next page.
5) The next page asks you to enter IP addresses to listen on
   incoming SMTP connections. Leave it as it is and Press ENTER
6) Even on the next page let the value be the default one and 
   Press ENTER.
7) Leave the next page as it is and Press ENTER.
8) The next page asks you for IP address or host name of the outgoing
   smarthost. Enter “smtp.example.com::587″. Where example refers to
   gmail, yahoo or any other mail service provider and 587 is port number.
9) The Next page asks you if you want to hide local mail name in 
   outgoing mail? Choose “No”.
10)The Next asks you if you want to keep number of DNS-queries minimal?
   Choose “No”.
11)On the next page choose the  delivery method for local mail as
   mbox format in /var/mail/.
12)Next page asks you if you want to split configuration into small
   files? Choose “No”. 
13)Next keep root and postmaster mail recipient empty.
\end{DoxyVerb}


Now terminal will show that M\-T\-A is being restarted.\par
 After this is done, run the following command in terminal \begin{DoxyVerb}$ sudo vim /etc/exim4/passwd.client
\end{DoxyVerb}


Add following in the file \begin{DoxyVerb}*:USERNAME@example.com:PASSWORD.

Where, USERNAME is  a valid email address and PASSWORD is  password for USERNAME.
\end{DoxyVerb}


7) Enable site(bakaplan) using Apache

First create symbolic link of bin(containing exe files) in html folder \begin{DoxyVerb}$ cd /path/to/bakaplan/src/html
$ ln -s ../cpp/bin ./cgi 
\end{DoxyVerb}


Do as following\-: \begin{DoxyVerb}1) Open misc/bakaplan.conf file
2) Change username(*) with your username and save
3) Copy bakaplan.conf to /etc/apache2/sites-available
    $ sudo cp ~/public_html/cgi-bin/bakaplan/misc/bakpalan.conf
    /etc/apache2/site-available
4) Enable site
    $ sudo a2ensite bakaplan.conf
5) Reload apache
    $ sudo service apache2 reload
6) Now open site on browser
    localhost/bakaplan
\end{DoxyVerb}


\section*{I\-N\-S\-T\-A\-L\-L\-A\-T\-I\-O\-N\-:}

Check \mbox{[}I\-N\-S\-T\-A\-L\-L\-A\-T\-I\-O\-N\mbox{]}() steps for using this software.

\section*{A\-U\-T\-H\-O\-R\-S\-:}

{\bfseries Mentor and Manager}

\href{https://github.com/hsrai}{\tt Dr. Hardeep Singh Rai}

Website\-: \href{http://gndec.ac.in/~hsrai}{\tt http\-://gndec.\-ac.\-in/$\sim$hsrai}

{\bfseries \href{https://github.com/GreatDevelopers/bakaplan/wiki/Contributors}{\tt Developers}}

\href{https://github.com/megha55}{\tt Mandeep Kaur} \href{https://github.com/inderpreetsingh}{\tt Inderpreet Singh} \href{https://github.com/Jaskaran28193}{\tt Jaskaran Singh}

Email\-: baithnekaplan \mbox{[}A\-T\mbox{]} gmail \mbox{[}D\-O\-T\mbox{]} com

This directory is for executable files

Source files

Input Files for generating seating plan

Output files for created while generating seating plan 