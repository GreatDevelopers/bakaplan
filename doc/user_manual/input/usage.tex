
BaKaPlan is developed in a way which is extremely easy to
understand and use as explained in this section.

\subsection {Registeration}
The last step of the installation (either manual or through script) will
lead you to the first page i.e Login. You will have to login everytime
 you want to use BaKaPlan.
\\
\image{0.3}{images/login.png}{Login Page}

As a first time user, you will have to register first. Clicking on 
\lq Register Here\rq, will direct you the page as shown in Figure 2.

\image{0.3}{images/register.png}{Register New User}

Here, enter your email id and click on \lq Register\rq.\\\\
This will display a new page with the message \lq Check verification mail
in your inbox (It might land in Junk/Spam folder and sometimes it may 
not be immediate).\rq\\
Now check your mail inbox and click on the hyperlink in the mail content 
which will direct you to Figure 3 where you will be asked to set password
 for your account.

\image{0.29}{images/passwd.png}{Set Password}

Once you have set your password, you will be again directed to Login Page.
 To get started enter your email id and password which you have just set. 

\subsection{Project Detail}
The next page (as shown in the figure below) asks for the name of the 
seating plan project you want to make.
 
\image{0.29}{images/projectDetail.png}{Project Detail}

Enter the name of your project and click on \lq Start New Project\rq.

\subsection{Branch Detail}

\image{0.3}{images/branchDetail.png}{Branch Detail}

This page asks for the information about the \textbf{Class Name} i.e. the name 
of the class whose exam is to be conduced, \textbf{Subject Name} related to the 
exam to be conducted and the corresponding \textbf{Subject Code}.\\

\textbf{NOTE}: For a single class, you can add names of more than one 
subject and their corresponding subject codes seperated by a comma.\\

The \textbf{Add Row} button allows you to add the information of another 
class. This feature allows you to make seating plan for a number of classes.\\
Similar to this is a \textbf{Delete Row} button in front of each row 
which will remove that row. 
\\

\subsection{Roll No. Detail}

The next page (Figure 6) is for providing the Roll Numbers for which 
the examination is tobe conducted.
The various fields to be provided on this page are :

\begin{itemize}
\item \textbf{Prefix}: This is the prefix value of the roll numbers 
specific to a class or branch.
\item \textbf{Start Roll no. \& End Roll. No}
\item \textbf{Not Included}: These are the Roll Numbers of the students
 who are not eligible for the examination. One can either give single values
 seperated by comma or and range of values(for eg. 29-34).\\\\\\
\end{itemize}
\image{0.3}{images/rollnoDetail.png}{Roll No. Detail}

\subsection{Datesheet}

\image{0.29}{images/datesheet.png}{Datesheet}

The \textbf{Datesheet} page (Figure 7) as the name suggests is for providing information
 about the dates on which the different examinations are to be held.\\
 The page asks for the \textbf{Date} of the exam and the corresponding
 \textbf{Exam Code} which is \textbf{same} as the \textbf{Subject Code}.\\
 The Add Row button can be used for setting exams on more than one date.\\
 At the end of the page is \lq Same room and exam details for each day\rq.
 Choosing \lq Yes\rq  will set the same exam detail for each date and 
 vice versa.
 
\subsection{Room Detail}
 
 In the \textbf{Room Detail} page we have to provide information about 
 the classes where the students appearing for the exam are to be seated.\\
 The page asks for the following two fields:
 
 \begin{itemize}
 \item \textbf{Centre Name}: Means the name of a block in the institute 
 constituting of a number of classes.
 \item \textbf{Room No.}: Means the name of the rooms being allocated
 for the examination in that centre. More than one rooms can be given in
  this space seperated by a comma. You need to provide the Room info 
 in the following manner - Room No:No. of Rows x No. of columns.\\
 NOTE: The symbol used in the above sentence is small \lq x\rq.\\
 \end{itemize}

\image{0.3}{images/roomDetail.png}{Room Detail} 
 
\subsection{Exam Detail}

\image{0.3}{images/examDetail.png}{}
 In the above given page, you need to fill the following fields:
\begin{itemize}
 \item \textbf{Name}: This is the name of the examination being carried 
 out for eg. sessionals or semester.
 \item \textbf{Session}: This is the session in which the examination
 would be carried out i.e. Morning or Evening.
 \item \textbf{Stat Time and End Time}: You can choose these timings
 from the drop down lists.
 \item \textbf{Venue}: For eg. Name of the College or University.
\end{itemize}

 
\subsection{Strategies}

BaKaPlan has five different strategies. Each strategy has its own
requirements, advantages and disadvantages. One should choose strategy
according to the needs and requirements.

List of strategies are as follows:

\begin{itemize}
\item Cushy
\item Serpentime
\item Flip-Flop
\item Triplet
\item Quadlet
\end{itemize}


\subsubsection{Cushy}
Cushy means easy. This is the simplest strategy. In this strategy,
seating plan is done continuously. For example, seating plan is done
by allocating all roll nos. of first subject code then continuous with
2nd subject code and so on.\\ \\
{\bf Requirement}
\begin{itemize}
\item Minimum 1 Class/Branch with minimum 1 subject code.
\end{itemize}
{\bf When to use}
\begin{itemize}
\item When there is only one subject code.
\item If examination has multiple sets of subjects.
\item If seats in examination has more space between them.
\item When there are equal seats with total students means there are no more
extra rooms/seats.
\item When only there is only one class for seating plan.
\end{itemize}
{\bf When not to use}
\begin{itemize}
\item When there is any entrance examination.
\item When there is no sets/codes for subjects.
\end{itemize}

\image{0.6}{images/cushy.png}{Cushy}

\subsubsection{Serpentine}
Serpentime is the simplest strategy. In this strategy,
seating plan is done continuously. For example, seating plan is done
by allocating all roll nos. of first subject code then continuous with
2nd subject code and so on.\\ \\
{\bf Requirement}
\begin{itemize}
\item Minimum 1 Class/Branch with minimum 1 subject code.
\end{itemize}
{\bf When to use}
\begin{itemize}
\item When there is only one subject code.
\item If examination has multiple sets of subjects.
\item If seats in examination has more space between them.
\item When there are equal seats with total students means there are no more
extra rooms/seats.
\item When only there is only one class for seating plan.
\end{itemize}
{\bf When not to use}
\begin{itemize}
\item When there is any entrance examination.
\item When there is no sets/codes for subjects.
\end{itemize}

\image{0.6}{images/serpentine.png}{Serpentine}


\subsubsection{Flip Flop}
Flip Flop is the second strategy. In this strategy, seating plan is
done by mixing two subject codes i.e. 1st subject code, 2nd subject
code, 1st subject code and so on.\\ \\
{\bf Requirement}
\begin{itemize}
\item Minimum 2 classes or 1 class with 2 subject codes.
\end{itemize}
{\bf When to use}
\begin{itemize}
\item When there are minimum 2 subject codes. 
\item When there is only one set for each subject in examination.
\end{itemize}

\image{0.6}{images/flipflop.png}{Flip Flop}


\subsubsection{Triplet}
Triplet means combination of three subject codes. Their is no seating plan
with continuous subject code.\\ \\
{\bf Requirement}
\begin{itemize}
\item Minimum 3 classes with 1 subject each or minimum 3 subject
codes.
\item Must have extra rooms if required.
\end{itemize}
{\bf When to use}
\begin{itemize}
\item When there are not sets or different codes for same subject
code.
\item When there are minimum three classes.
\item When extra rooms are available.
\end{itemize}

\image{0.6}{images/triplet.png}{Triplet}


\subsubsection{Quadlet}
Quadlet means combination of four subject codes. It is used whem heir 
is no seating plan with continuous subject code.\\ \\
{\bf Requrement}
\begin{itemize}
\item Minimum four classes with 1 subject each or minimum 3 subject
codes.
\item Must have extra rooms if required.
\end{itemize}
{\bf When to use}
\begin{itemize}
\item When there are not sets or different codes for same subject
code.
\item When there are minimum four classes.
\item When extra rooms are available.
\item For entrance test.
\item When their is less distance between rows and columns.
\end{itemize}

\image{0.6}{images/quadlet.png}{Quadlet}


\subsection{Valid Strategy Detail}
This page gives the final details of the information given by the user 
and tells if the given number of rooms are sufficient according to the 
number of students to be seated.

\image {0.3}{images/validSrategy2.png}{Valid Strategy Detail}


\subsection{Seating Plan}
This is the final page.
The user is provided with 2 types of views - {\bf HTML} and {\bf PDF} 
views.
\image {0.3}{images/seatingPlan.png}{Seating Plan}
